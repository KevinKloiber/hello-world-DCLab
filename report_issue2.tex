\documentclass{article}
\usepackage{listings}
\usepackage{graphicx}
\usepackage[legalpaper, margin=1in]{geometry}
\lstset{frame=tb,
  language=Java,
  aboveskip=3mm,
  belowskip=3mm,
  showstringspaces=false,
  columns=flexible,
  basicstyle={\small\ttfamily},
  numbers=none,
  numberstyle=\tiny\color{gray},
  keywordstyle=\color{blue},
  commentstyle=\color{dkgreen},
  stringstyle=\color{mauve},
  breaklines=true,
  breakatwhitespace=true,
  tabsize=3
}

\usepackage{ifxetex,ifluatex}
\ifxetex
  \usepackage{polyglossia}
  \setmainlanguage{english}
  \usepackage{fontspec}
\else\ifluatex
  \usepackage{polyglossia}
  \setmainlanguage{english}
  \usepackage{fontspec}
\else
  \usepackage[english]{babel}
  \usepackage[utf8]{inputenc}
  \usepackage[T1]{fontenc}
  \usepackage{lmodern}
\fi\fi

\usepackage[
  hashEnumerators,
  definitionLists,
  footnotes,
  inlineFootnotes,
  smartEllipses,
  fencedCode,
  contentBlocks
]{markdown}

\begin{document}
\title{Balance on CPT/ICD in ambulance claims}
\begin{enumerate}
	\item Procedure
	\begin{itemize}
		\item I used the diagnosis code that I found in the ambulance claims to merge it into the main analysis dataset we are using right now. I am able to match 100\% of the observations.
		\item I created 3-digit diagnosis codes, the same way we did in the main dataset. I dropped 'weird' ambulance diagnosis codes such as XX0, X00, XXX, 86, A and //
		\item 17.17 \% of the ambulance ICD codes are equal to the diagnosis code we have in the main dataset, 82.83 \% are not!
		\item I regressed our main mortality variable, ```y\_mortality\_28``` on ambulance ICD code dummies, absorbing ambulance FE. Then I predicted mortality, using the linear option xb.
		\begin{lstlisting}
		areg y_mortality_28 i.diag3_amb, a(z_tax_num)
		predict mort_hat, xb
		\end{lstlisting}
		\item I residualized the instrument in the same way we did before, using race, age, date and utilization (our X1) with zip code FE.

		\begin{lstlisting}
		xi: areg z_lova x1, absorb(z_zip_5)
		predict rez_z_lova, resid
		\end{lstlisting}
	\end{itemize}
	\newpage
	\item Results

	\begin{itemize}
		\item Binscatter of predicted mortality and residualized ambulance instrument (lova)
	\end{itemize}
		\includegraphics{49673907-fa4e4400-fa24-11e8-9e47-8dfc61811d85.png}
	
	\begin{itemize}
		\item Plot of y\_hat for ambulance ICD codes, absorbing ambulance FE, and 28-Day actual mortality on instrument controlling for our new x1.
	\end{itemize}
	\includegraphics{binscatter.png}
	\newpage
	\item Quick analysis to describe what are the top 30 ICDs that are used as ambulance ICDs but are not in the main dataset
	\begin{markdown}

	diag3_ambul |
           ance |      Freq.     Percent        Cum.

            786 |    141,539       22.41       22.41
            780 |    132,795       21.02       43.43
       		959 |     31,465        4.98       48.41
        	789 |     31,140        4.93       53.34
        	719 |     19,653        3.11       56.45
        	787 |     17,877        2.83       59.28
	        729 |     11,818        1.87       61.15
	        724 |     11,442        1.81       62.96
	        436 |     11,022        1.74       64.71
	        427 |     10,205        1.62       66.32
	        796 |      8,886        1.41       67.73
	        784 |      8,683        1.37       69.10
	        799 |      7,820        1.24       70.34
	        518 |      7,076        1.12       71.46
	        459 |      7,008        1.11       72.57
	        790 |      6,380        1.01       73.58
	        785 |      5,914        0.94       74.52
	        298 |      5,827        0.92       75.44
	        829 |      5,139        0.81       76.25
	        578 |      5,036        0.80       77.05
	        401 |      5,017        0.79       77.84
	        458 |      4,562        0.72       78.57
	        250 |      4,247        0.67       79.24
	        996 |      4,192        0.66       79.90
	        781 |      4,090        0.65       80.55
	        873 |      3,978        0.63       81.18
	        869 |      3,948        0.62       81.80
	        V47 |      3,921        0.62       82.42
	        782 |      3,752        0.59       83.02
	        A99 |      3,619        0.57       83.59
	        251 |      3,433        0.54       84.13
	        300 |      3,419        0.54       84.68
\end{markdown}
\begin{markdown}

I don't know how fluent you are in ICD codes so I translated the TOP 10:

	1. Dyspnea and respiratory abnormalities
	2. General symptoms
	3. Other and unspecified injury to head face and neck
	4. Abdonimal pain / Other symptoms involving abdomen and pelvis
	5. Palindromic rheumatism 
	6. Dysphagia
	7. Nontraumatic compartment syndrome / swelling of limb
	8. Other and unspecified disorders of back
	9. Acute, but ill-defined, cerebrovascular disease
	10. Other specified cardiac dysrhythmias

I think that sounds pretty reasonable. Most of the symptoms are fairly vague because you don't have that much time in the ambulance. 1. could turn into PNA or some other respiratory abnormalities. 3. could be patients that were transported from an acute event to the hospital, and also the other stuff looks 'convincing', do you agree?

Here's also the TOP 10 of 5 digit ICD codes:

	ICD 5	Freq.	Name
	786.09	13.3	Other respiratory abnormalities
	786.50	12.8%	Chest pain, unspecified
	786.05	11.2%	Shortness of breath
	780.79	7.4%	Other malaise and fatigue
	780.09	6.4%	Other alteration of consciousness
	780.02	6.0%	Transient alteration of awareness
	078.02	4.6%	Syncope and collapse
	789.00	4.5%	Abdominal pain, unspecified site
	780.97	3.9%	Altered mental status
	078.04	3.7%	Dizziness and giddiness

All in all, seems like plausible reasons to call an ambulance while not having super specific diagnostic codes.
\end{markdown}
\end{enumerate}

\end{document}