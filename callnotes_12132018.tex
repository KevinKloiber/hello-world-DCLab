\documentclass{article}
\usepackage{listings}
\usepackage{graphicx}
\usepackage[legalpaper, margin=1in]{geometry}
\lstset{frame=tb,
  language=Java,
  aboveskip=3mm,
  belowskip=3mm,
  showstringspaces=false,
  columns=flexible,
  basicstyle={\small\ttfamily},
  numbers=none,
  numberstyle=\tiny\color{gray},
  keywordstyle=\color{blue},
  commentstyle=\color{dkgreen},
  stringstyle=\color{mauve},
  breaklines=true,
  breakatwhitespace=true,
  tabsize=3
}

\usepackage{ifxetex,ifluatex}
\ifxetex
  \usepackage{polyglossia}
  \setmainlanguage{english}
  \usepackage{fontspec}
\else\ifluatex
  \usepackage{polyglossia}
  \setmainlanguage{english}
  \usepackage{fontspec}
\else
  \usepackage[english]{babel}
  \usepackage[utf8]{inputenc}
  \usepackage[T1]{fontenc}
  \usepackage{lmodern}
\fi\fi

\usepackage[
  hashEnumerators,
  definitionLists,
  footnotes,
  inlineFootnotes,
  smartEllipses,
  fencedCode,
  contentBlocks
]{markdown}

\begin{document}
\title{Balance on CPT/ICD in ambulance claims}
\section{Balance on CPT/ICD in ambulance claims}

\begin{itemize}
    \item Diagnosis codes found in the ambulance claims.
    \item 17.17 \% of the ambulance ICD codes at the 3 digit level are equal to the diagnosis code we have in the main dataset, 82.83 \% are not.
    \item Regression of ```y\_mortality\_28``` on ambulance ICD code dummies, absorbing ambulance FE.
    \item Binscatter of predicted mortality and residualized ambulance instrument (lova)
    \end{itemize}
    \includegraphics{49673907-fa4e4400-fa24-11e8-9e47-8dfc61811d85.png}
\begin{itemize}
    \item Plot of y\_hat for ambulance ICD codes, absorbing ambulance FE, and 28-Day actual mortality on instrument controlling for our new x1.
    \end{itemize}
    \includegraphics{binscatter.png}
    \newpage
\section{Analysis of Differences in ICD codes}

Quick analysis to describe what are the top 30 ICDs that are used as ambulance ICDs but are not in the main dataset
  \begin{markdown}

  diag3_ambul |
         ance |      Freq.     Percent        Cum.

          786 |    141,539       22.41       22.41
          780 |    132,795       21.02       43.43
          959 |     31,465        4.98       48.41
          789 |     31,140        4.93       53.34
          719 |     19,653        3.11       56.45
          787 |     17,877        2.83       59.28
          729 |     11,818        1.87       61.15
          724 |     11,442        1.81       62.96
          436 |     11,022        1.74       64.71
          427 |     10,205        1.62       66.32

\end{markdown}
\begin{markdown}

Here is the list in english:

    1. Dyspnea and respiratory abnormalities
    2. General symptoms
    3. Other and unspecified injury to head face and neck
    4. Abdonimal pain / Other symptoms involving abdomen and pelvis
    5. Palindromic rheumatism 
    6. Dysphagia
    7. Nontraumatic compartment syndrome / swelling of limb
    8. Other and unspecified disorders of back
    9. Acute, but ill-defined, cerebrovascular disease
    10. Other specified cardiac dysrhythmias


Here's also the TOP 10 of 5 digit ICD codes:

    ICD 5   Freq. Name
    786.09  13.3  Other respiratory abnormalities
    786.50  12.8% Chest pain, unspecified
    786.05  11.2% Shortness of breath
    780.79  7.4%  Other malaise and fatigue
    780.09  6.4%  Other alteration of consciousness
    780.02  6.0%  Transient alteration of awareness
    078.02  4.6%  Syncope and collapse
    789.00  4.5%  Abdominal pain, unspecified site
    780.97  3.9%  Altered mental status
    078.04  3.7%  Dizziness and giddiness

All in all, seems like plausible reasons to call an ambulance while not having super specific diagnostic codes.
\end{markdown}

\end{document}